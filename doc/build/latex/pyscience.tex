%% Generated by Sphinx.
\def\sphinxdocclass{report}
\documentclass[letterpaper,10pt,english]{sphinxmanual}
\ifdefined\pdfpxdimen
   \let\sphinxpxdimen\pdfpxdimen\else\newdimen\sphinxpxdimen
\fi \sphinxpxdimen=.75bp\relax

\PassOptionsToPackage{warn}{textcomp}
\usepackage[utf8]{inputenc}
\ifdefined\DeclareUnicodeCharacter
% support both utf8 and utf8x syntaxes
\edef\sphinxdqmaybe{\ifdefined\DeclareUnicodeCharacterAsOptional\string"\fi}
  \DeclareUnicodeCharacter{\sphinxdqmaybe00A0}{\nobreakspace}
  \DeclareUnicodeCharacter{\sphinxdqmaybe2500}{\sphinxunichar{2500}}
  \DeclareUnicodeCharacter{\sphinxdqmaybe2502}{\sphinxunichar{2502}}
  \DeclareUnicodeCharacter{\sphinxdqmaybe2514}{\sphinxunichar{2514}}
  \DeclareUnicodeCharacter{\sphinxdqmaybe251C}{\sphinxunichar{251C}}
  \DeclareUnicodeCharacter{\sphinxdqmaybe2572}{\textbackslash}
\fi
\usepackage{cmap}
\usepackage[T1]{fontenc}
\usepackage{amsmath,amssymb,amstext}
\usepackage{babel}
\usepackage{times}
\usepackage[Sonny]{fncychap}
\ChNameVar{\Large\normalfont\sffamily}
\ChTitleVar{\Large\normalfont\sffamily}
\usepackage{sphinx}

\fvset{fontsize=\small}
\usepackage{geometry}

% Include hyperref last.
\usepackage{hyperref}
% Fix anchor placement for figures with captions.
\usepackage{hypcap}% it must be loaded after hyperref.
% Set up styles of URL: it should be placed after hyperref.
\urlstyle{same}
\addto\captionsenglish{\renewcommand{\contentsname}{Contents:}}

\addto\captionsenglish{\renewcommand{\figurename}{Fig.\@ }}
\makeatletter
\def\fnum@figure{\figurename\thefigure{}}
\makeatother
\addto\captionsenglish{\renewcommand{\tablename}{Table }}
\makeatletter
\def\fnum@table{\tablename\thetable{}}
\makeatother
\addto\captionsenglish{\renewcommand{\literalblockname}{Listing}}

\addto\captionsenglish{\renewcommand{\literalblockcontinuedname}{continued from previous page}}
\addto\captionsenglish{\renewcommand{\literalblockcontinuesname}{continues on next page}}
\addto\captionsenglish{\renewcommand{\sphinxnonalphabeticalgroupname}{Non-alphabetical}}
\addto\captionsenglish{\renewcommand{\sphinxsymbolsname}{Symbols}}
\addto\captionsenglish{\renewcommand{\sphinxnumbersname}{Numbers}}

\addto\extrasenglish{\def\pageautorefname{page}}

\setcounter{tocdepth}{1}



\title{pyscience Documentation}
\date{Feb 19, 2019}
\release{0.1.0.dev2}
\author{Manuel Alcaraz}
\newcommand{\sphinxlogo}{\vbox{}}
\renewcommand{\releasename}{Release}
\makeindex
\begin{document}

\ifdefined\shorthandoff
  \ifnum\catcode`\=\string=\active\shorthandoff{=}\fi
  \ifnum\catcode`\"=\active\shorthandoff{"}\fi
\fi

\pagestyle{empty}
\sphinxmaketitle
\pagestyle{plain}
\sphinxtableofcontents
\pagestyle{normal}
\phantomsection\label{\detokenize{index::doc}}


Pyscience is a easy python library and command-line application to work with
mathematical operations and other science related programming. Also you can manage
CSV data with the datam module. See API documentation for more information.

Pyscience requires Python 3.7+ to work.


\chapter{Pyscience installation}
\label{\detokenize{installation:pyscience-installation}}\label{\detokenize{installation::doc}}
Pyscience needs a recent version of python (3.7). You can download it \sphinxhref{https://www.python.org/downloads/}{here}. It won’t work in older versions.
Pyscience is available in pip. To install, type with superuser privileges:

\begin{sphinxVerbatim}[commandchars=\\\{\}]
\PYG{n}{pip3} \PYG{n}{install} \PYG{n}{pyscience}
\end{sphinxVerbatim}

If you want to install the latest development version, use the github repo:

\begin{sphinxVerbatim}[commandchars=\\\{\}]
\PYG{n}{GitHub} \PYG{n}{repository} \PYG{o+ow}{is} \PYG{o+ow}{not} \PYG{n}{yet} \PYG{n}{available}
\end{sphinxVerbatim}


\chapter{User guide}
\label{\detokenize{user_guide:user-guide}}\label{\detokenize{user_guide::doc}}
Pyscience’s interpreter is very easy to use. Start it with:

\begin{sphinxVerbatim}[commandchars=\\\{\}]
\PYG{n}{pyscience}
\end{sphinxVerbatim}


\section{Working as a calculator}
\label{\detokenize{user_guide:working-as-a-calculator}}
Pyscience uses python \sphinxcode{\sphinxupquote{eval}} function to evaluate expressions after expand it.
You can use pyscience as a normal calculator:

\begin{sphinxVerbatim}[commandchars=\\\{\}]
\PYG{o}{\PYGZgt{}} \PYG{l+m+mi}{2} \PYG{o}{+} \PYG{l+m+mi}{2}
\PYG{l+m+mi}{4}
\PYG{o}{\PYGZgt{}} \PYG{l+m+mi}{3} \PYG{o}{*} \PYG{p}{(}\PYG{l+m+mi}{2} \PYG{o}{+} \PYG{l+m+mi}{3}\PYG{p}{)}
\PYG{l+m+mi}{15}
\PYG{o}{\PYGZgt{}} \PYG{l+m+mi}{3}\PYG{p}{(}\PYG{l+m+mi}{3}\PYG{o}{+}\PYG{l+m+mi}{4}\PYG{p}{)}
\PYG{l+m+mi}{7}
\end{sphinxVerbatim}


\subsection{Addition}
\label{\detokenize{user_guide:addition}}
To add two numbers, use the \sphinxcode{\sphinxupquote{+}} operator:

\begin{sphinxVerbatim}[commandchars=\\\{\}]
\PYG{o}{\PYGZgt{}} \PYG{l+m+mi}{2} \PYG{o}{+} \PYG{l+m+mi}{3}
\PYG{l+m+mi}{5}
\end{sphinxVerbatim}


\subsection{Subtraction}
\label{\detokenize{user_guide:subtraction}}
To subtract two number, use the \sphinxcode{\sphinxupquote{-}} operator:

\begin{sphinxVerbatim}[commandchars=\\\{\}]
\PYGZgt{} 2 \textendash{} 3
\PYGZhy{}1
\end{sphinxVerbatim}


\subsection{Multiplication}
\label{\detokenize{user_guide:multiplication}}
To multiply two numbers, use the \sphinxcode{\sphinxupquote{*}} operator:

\begin{sphinxVerbatim}[commandchars=\\\{\}]
\PYG{o}{\PYGZgt{}} \PYG{l+m+mi}{2} \PYG{o}{*} \PYG{l+m+mi}{3}
\PYG{l+m+mi}{6}
\end{sphinxVerbatim}


\subsection{Division}
\label{\detokenize{user_guide:division}}
To divide one number by other, use the \sphinxcode{\sphinxupquote{/}} operator:

\begin{sphinxVerbatim}[commandchars=\\\{\}]
\PYG{o}{\PYGZgt{}} \PYG{l+m+mi}{8} \PYG{o}{/} \PYG{l+m+mi}{2}
\PYG{l+m+mi}{4}
\end{sphinxVerbatim}


\subsection{Powers}
\label{\detokenize{user_guide:powers}}
You can create powers using the \sphinxcode{\sphinxupquote{**}} operator:

\begin{sphinxVerbatim}[commandchars=\\\{\}]
\PYG{o}{\PYGZgt{}} \PYG{l+m+mi}{2} \PYG{o}{*}\PYG{o}{*} \PYG{l+m+mi}{4}
\PYG{l+m+mi}{16}
\end{sphinxVerbatim}

or using \textasciicircum{} numbers:

\begin{sphinxVerbatim}[commandchars=\\\{\}]
\PYGZgt{} 2\(\sp{\text{4}}\)
16
\end{sphinxVerbatim}


\subsection{Fractions}
\label{\detokenize{user_guide:fractions}}
The \sphinxcode{\sphinxupquote{F}} (\sphinxcode{\sphinxupquote{Fraction}}) class provides a way to create and operate with fractions.
Numerator and denominator are divided using a coma. For example:

\begin{sphinxVerbatim}[commandchars=\\\{\}]
\PYG{o}{\PYGZgt{}} \PYG{n}{F}\PYG{p}{(}\PYG{l+m+mi}{2}\PYG{p}{,}\PYG{l+m+mi}{3}\PYG{p}{)} \PYG{o}{+} \PYG{n}{F}\PYG{p}{(}\PYG{l+m+mi}{3}\PYG{p}{,}\PYG{l+m+mi}{4}\PYG{p}{)}
\PYG{n}{F}\PYG{p}{(}\PYG{l+m+mi}{17}\PYG{p}{,}\PYG{l+m+mi}{12}\PYG{p}{)}
\end{sphinxVerbatim}

You can use the same operators for fractions.


\section{Working with algebra}
\label{\detokenize{user_guide:working-with-algebra}}
Pyscience can operate with Monomials, Variables and Polynomials. Some examples of
what can you do:

\begin{sphinxVerbatim}[commandchars=\\\{\}]
\PYG{o}{\PYGZgt{}} \PYG{l+m+mi}{2}\PYG{n}{x} \PYG{o}{+} \PYG{l+m+mi}{3}\PYG{n}{x}
\PYG{l+m+mi}{5}\PYG{n}{x}
\PYG{o}{\PYGZgt{}} \PYG{l+m+mi}{3}\PYG{n}{x} \PYG{o}{*} \PYG{l+m+mi}{6}\PYG{n}{y}
\PYG{l+m+mi}{18}\PYG{n}{xy}
\PYG{o}{\PYGZgt{}} \PYG{l+m+mi}{2}\PYG{n}{x} \PYG{o}{/} \PYG{p}{(}\PYG{l+m+mi}{2}\PYG{n}{x}\PYG{p}{)}
\PYG{l+m+mi}{1}
\end{sphinxVerbatim}

\begin{sphinxadmonition}{note}{Note:}
In the last example, you can think why I have put parenthesis for the second Monomial. If you don’t do it, you will divide \sphinxstyleemphasis{2x} by \sphinxstyleemphasis{2} and, AFTER, you will multiply the result by \sphinxstyleemphasis{x}. In this case, the final result is \sphinxstyleemphasis{x\(\sp{\text{2}}\)}
\end{sphinxadmonition}


\section{Working with chemical elements}
\label{\detokenize{user_guide:working-with-chemical-elements}}
Pyscience can show you basic information about chemical elements. You can do it
with the \sphinxcode{\sphinxupquote{CE}} function:

\begin{sphinxVerbatim}[commandchars=\\\{\}]
\PYGZgt{} CE(‘H’)
...
\end{sphinxVerbatim}

If you want to set a specific mass for the element, indicate that between brackets:

\begin{sphinxVerbatim}[commandchars=\\\{\}]
\PYGZgt{} CE(‘Si(32)’) \PYGZsh{} Set mass to 32
...
\end{sphinxVerbatim}

Also, you can work with elements which have charge:

\begin{sphinxVerbatim}[commandchars=\\\{\}]
\PYGZgt{} CE(‘Si2+’)
...
\end{sphinxVerbatim}

If you know the atomic number of a element but not the symbol, you can get the
element by its atomic number:

\begin{sphinxVerbatim}[commandchars=\\\{\}]
\PYG{o}{\PYGZgt{}} \PYG{n}{CE}\PYG{p}{(}\PYG{l+m+mi}{20}\PYG{p}{)}
\PYG{c+c1}{\PYGZsh{} Calcium (Ca)}
\end{sphinxVerbatim}


\chapter{License}
\label{\detokenize{license:license}}\label{\detokenize{license::doc}}
pyscience - python science programming

Copyright (c) 2019 Manuel Alcaraz Zambrano

Permission is hereby granted, free of charge, to any person obtaining a copy
of this software and associated documentation files (the “Software”), to deal
in the Software without restriction, including without limitation the rights
to use, copy, modify, merge, publish, distribute, sublicense, and/or sell
copies of the Software, and to permit persons to whom the Software is
furnished to do so, subject to the following conditions:

The above copyright notice and this permission notice shall be included in all
copies or substantial portions of the Software.

THE SOFTWARE IS PROVIDED “AS IS”, WITHOUT WARRANTY OF ANY KIND, EXPRESS OR
IMPLIED, INCLUDING BUT NOT LIMITED TO THE WARRANTIES OF MERCHANTABILITY,
FITNESS FOR A PARTICULAR PURPOSE AND NONINFRINGEMENT. IN NO EVENT SHALL THE
AUTHORS OR COPYRIGHT HOLDERS BE LIABLE FOR ANY CLAIM, DAMAGES OR OTHER
LIABILITY, WHETHER IN AN ACTION OF CONTRACT, TORT OR OTHERWISE, ARISING FROM,
OUT OF OR IN CONNECTION WITH THE SOFTWARE OR THE USE OR OTHER DEALINGS IN THE
SOFTWARE.


\chapter{Indices and tables}
\label{\detokenize{index:indices-and-tables}}\begin{itemize}
\item {} 
\DUrole{xref,std,std-ref}{genindex}

\item {} 
\DUrole{xref,std,std-ref}{modindex}

\item {} 
\DUrole{xref,std,std-ref}{search}

\end{itemize}



\renewcommand{\indexname}{Index}
\printindex
\end{document}